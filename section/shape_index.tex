\section{Shape index and query shape containment}

We define a Shape Index (SI) as a set of mappings between RDF data shapes and sets of resources.
Additionally, an SI has an associated domain of action
and a flag indicating if each resource in the domain is bound by a shape. 
We refer to a SI with this flag as a \emph{complete} Shape Index.
In a SI when a shape is in relation to a set of RDF resources then the shape must validate them.
Furthermore, every set of triples respecting the shape in the domain must be located inside one resource of the set.

Our approach consists of determining before the traversal of a whole domain the location of the useful resources for the query execution.
The query engine starts with a "maximalist" lookup policy and adapts it to a more restrictive one upon the evaluation of the SI.
For that purpose, the query engine must first discover the SI in each (sub)domain.
In the case of Solid, the SI should be at the root of the storage to be easily discoverable.
The evaluation of the query with the SI is similar to the classic query containment problem.
Indeed, we propose that we can transform a shape into a query ($Q_{s}$).
In this short paper, we don't provide proof for this proposition, however, 
\citeauthor{Delva2021} give an intuition by demonstrating how to query RDF subgraphs using RDF data shapes as a query language.
We divide the query into multiple star patterns with their dependencies ($Q_{star}$).
The evaluation of the SI consists of finding if those $Q_{star}$ are contained inside the shapes of the SI.
If all the $Q_{star}$ are contained in a $Q_{s}$ or have no binding with any $Q_{s}$ of the SI
then we adapt the lookup policy to ignore all the other resources inside the domain even if the SI is \emph{incomplete}.
If the SI is \emph{complete} and not all the $Q_{star}$ are contained in a $Q_{s}$ then the lookup policy can be adapted
to visit every resource in relation to a $Q_{s}$ with a partial binding with a $Q_{star}$.
