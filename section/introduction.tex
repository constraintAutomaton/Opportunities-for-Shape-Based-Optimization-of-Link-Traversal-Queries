% TeX root = ../Main.tex

\section{Introduction}
 
The large-scale publication of linked data empowers more freedom in the creation and usage of web applications.
More concretely linked data can diminish data silos \cite{Verstraete2022}
and could foster potential new forms of application ownership \cite{Mechant2021}.
The \href{https://solidproject.org/TR/protocol}{Solid} and
\href{https://docs.joinmastodon.org/}{Mastodon} protocols
are popular technologies utilizing the emancipatory potential of linked data.
Linked data is mostly modeled and published using the graph data formalization \href{https://www.w3.org/TR/rdf12-concepts/}{RDF} and its serializations.
RDF Terms can be expressed using IRIs, Blank Nodes, and Litterals.
The usage of IRIs provides more reusability of data and explicitizes in a machine-interpretable way the relations between
information from multiple remote or local subgraphs.
More information about an IRI term can be found by dereferencing them using the HTTP protocol.\rt{Not all IRIs can be dereferenced, but all URLs can.}

\rt{Make sure to cite a paper from Alberto Mendelzon in the next paragraph. You can look at Taelman2023 for an example, as I cite one there as well.}
Link Traversal Query Processing (LTQP) \cite{Hartig2012} is a query paradigm that exploits
\rt{You go into the details of LTQP too quickly. First start by explaining its motivation (querying a globally distributed dataspace without prior indexing), and constrast it to traditional centralized querying}
the informative potential of IRIs dereferencing during SPARQL queries.
LTQP starts with a few URIs called seed URIs and recursively dereferences them from an internal data source following a lookup policy.
While LTQP enables live exploration of environments without prior indexing, it leads to some difficulties.
One of them is the pseudo-infinite search domain derived from the size of the World Wide Web \cite{Hartig2014}.
Additionally, HTTP requests can be very slow and unpredictable making their execution the bottleneck of the method \cite{hartig2016walking}.
Reachability criteria \cite{Hartig2012} are a partial answer to this problem by defining completeness on the traversal of URIs
contained in the internal data source instead of on the acquisition of all the results or the traversal of the whole web.
Those criteria can also be used as a lookup policy for dereferencing of external data sources.
Another difficulty is the lack of a priori information about the sources rendering query planning arduous.
To alleviate this problem, the current state of the art is to use carefully crafted heuristics for joins ordering \cite{Hartig2011}.
Those heuristics provide non-optimal fairly performant query plans.
The limitations of this approach are usually of little importance because the main bottleneck is the high number of HTTP requests.
In response to this, current research focuses on providing fast results to the user by ordering adequately the dereferencing operations of IRIs \cite{hartig2016walking}.

LTQP research considers that resources are served by HTTP servers.
However, some subsets of the web are expose by web servers implementing protocols above the application layer of the TCP/IP stack.
Those protocols organize the location of the information published in an HTTP server.
An examples of them is Solid.
We refer to those subsets of the web as Link Data Structured Environments (LDSE).\rt{Not sure it makes sense to consider this as a subset. All Linked Data environments have some structure, but each environment may have different structures. I would suggest just pointing out this fact, and that you can then draw assumptions over them.}
An important aspect of LDSE is the explicit structural assumptions.
Structural assumptions act as contracts between the data provider and 
the query engines stipulating that in a certain subdomain of the web, some information respecting a constraint can be found.
Those assumptions can be declared in the form of hypermedia descriptions \cite{Fielding}.
It has been shown that by making query engines exploit those assumptions it is possible to reduce the query execution time
of realistic queries to the extent where the bottleneck is not the execution of 
HTTP requests but the suboptimal heuristic-based query plan \cite{eschauzier_quweda_2023, Taelman2023}.
Yet for multiple queries, the bottleneck remains the high number of HTTP requests  \cite{eschauzier_quweda_2023}.
It is reasonable to hypothesize that a significant portion of those HTTP requests lead to the dereferencing of
documents containing data that don't contribute to the result of the query.
Hence investigating more descriptive structural assumptions is a relevant research endeavor.
\rt{I recommend talking about link pruning somewhere around here and guided link traversal: https:\/\/www.rubensworks.net\/publications\/verborgh\_amw\_2020\/}
Structural assumptions based on the path structure of URIs are not a viable solution. \rt{But this is what you're doing, right? You link shapes to URL paths.}
Indeed, those types of assumptions either don't describe enough the data or are too exhaustive to be effective and practical.
Furthermore, the semantics of the structure of the path of URI is not easily machine-interpretable, hence \rt{This is confusing, because you are literally attaching semantics using shapes.}
making this approach not suitable for a generalizable solution.

In this article, we propose to use RDF shapes as the main mechanism for a structural assumption in the form of a Shape Index (SI). \rt{Contrast this to the Solid type index somewhere?}
RDF shapes are mostly used in data validation \cite{Gayo2018a} hence they provide an excellent means to describe data.
They are also used to a lesser extent in the context of federated query optimization \cite{kashif2021}.
RDF shapes don't require significative power to be maintained because they only need to be edited 
when the data model is modified.
The low cost of maintenance combined with the mostly passive contribution of 
the server when using RDF shape for query optimization makes it a promising potential medium. 
We foresee opportunities for using a shape index during data source discovery, link pruning and ordering, and query planning.
This short paper present our preliminary work on data discovery and link pruning.

\rt{I'm surprised there is no example on how shapes can help in a Solid pod. I would recommend to add an intuitive example before you introduce shapes above. You could give the example of finding posts, and thereby omitting data from other shapes.}
